\documentclass[a4paper]{article}

\usepackage[utf8]{inputenc}
\usepackage[T1]{fontenc}
\usepackage{amsmath}
\usepackage{graphicx}
\usepackage{fancyhdr}
\usepackage{tabularx}
\usepackage{hyperref}
\usepackage{etoolbox}% http://ctan.org/pkg/etoolbox
\usepackage{needspace}% http://ctan.org/pkg/needspace

% Für Theoreme (Beispiele, Definitionen, Sätze etc.)
\usepackage{amsthm}

\usepackage[colorinlistoftodos,prependcaption,textsize=tiny]{todonotes}

%\Bearbeitet Format, sodass statt der Einrückung bei Paragraphen durch Zeilenabstand ersetzt wird
\usepackage{parskip}
\setlength\parindent{0pt}

%\Setzt Dokumentinformationen
\title{Master-Thesis: Incremental Analysis of Software Product Lines}
\author{Moritz Fl\"oter}
\date{June 2018}



\begin{document}
\bibliographystyle{unsrt}

\newtheoremstyle{mystyle}% name
  {3pt}%Space above
  {3pt}%Space below
  {\normalfont}%Body font
  {0pt}%Indent amount
  {\bfseries}% Theorem head font
  {}%Punctuation after theorem head
  {\newline}%Space after theorem head 2
  {}%Theorem head spec (can be left empty, meaning 'normal')

\theoremstyle{mystyle}


\newtheorem{req}{REQ}
\newtheorem{subreq}{REQ}[req]
\newtheorem{subsubreq}{REQ}[req]
\AtBeginEnvironment{req}{\Needspace{5\baselineskip}}
\setcounter{req}{0}

\newcommand*{\reqtable}[4]{
\begin{tabular}{ | p{0.15\textwidth} | p{0.79\textwidth} | }
	\hline
	\textit{Priority} & \begin{minipage}[l]{0.79\textwidth}
	\vspace{0.25em}
		#1
	\vspace{0.25em}
	\end{minipage} \\ \hline
	\textit{Source} & \begin{minipage}[l]{0.79\textwidth}
	\vspace{0.25em}
		#2
	\vspace{0.25em}
	\end{minipage}\\ \hline
	\textit{Description} & \begin{minipage}[l]{0.79\textwidth}
	\vspace{0.25em}
		#3
	\vspace{0.25em}
	\end{minipage} \\ \hline
	\textit{Explanation} & \begin{minipage}[l]{0.79\textwidth} 
	\vspace{0.25em}
		#4
	\vspace{0.25em}
	\end{minipage} \\
	\hline
\end{tabular}
}



%\Deckblatt
\maketitle
\newblock

\begin{center}
floeter@uni-hildesheim.de \par
Matrikel-Nr: 236278 \par
Supervisor: \par
Prof. Dr. Klaus Schmid \par
Christian K\"oher
\end{center}

\newpage
\lhead{{}}
\rhead{\leftmark}
\pagestyle{fancy}

\listoftodos[Notes]
\clearpage

%\Inhaltsverzeichnis
\tableofcontents
\newpage
\bibliographystyle{unsrt}

%\Deckblatt
\maketitle
\newpage

\setcounter{page}{1}
\lhead{{}}
\rhead{\leftmark}
\pagestyle{fancy}



\section{Introduction}

\todo{Nicht vergessen: Aufbau der Arbeit erneut beschreiben}

\section{Background}
\subsection{Software Product Lines}

\subsection{Software Product Line Analyses}

\subsection{KernelHaven - an Analysis Infrastructure}

\subsection{Objective for this Work}


\clearpage
\section{Requirements for the Software System}

\begin{req}[Working Base of Incremental Analysis]
	\reqtable
	{Must}  {Interview}
	{The incremental analysis must work based on the current state of a repository and a proposed change}
	{The working base of each incremental analysis is the previous state before a commit/code change. The increment is represented by the change introduced.}
	
	\begin{subreq}[Input Format for Incremental Analysis] \label{req:git-diff}
		\reqtable
		{Must}  {Interview}
		{The input must be a git-diff. Furthermore a filebase upon which the diff can be applied is required.}
		{The main input for the analysis is a git-diff. This diff represents a changeset that is to be analysed.}
	\end{subreq}
\end{req}

\begin{req}[Covered Analysis-Types]
	\reqtable
	{Must}  {Initial Description, Interview}
	{The incremental analysis must support block-based analyses.}
	{}
	
	\begin{subreq}[Input Format for Incremental Analysis] \label{req:git-diff}
		\reqtable
		{Must}  {Interview}
		{The input must be a git-diff}
		{The main input for the analysis is a git-diff. This diff represents a changeset that is to be analysed.}
	\end{subreq}
\end{req}


\begin{req}[Filtering of Source Files] \label{req:early-filtering}
\reqtable
	{Must}  {Interview}
	{The set of source files upon which the analysis operates should be reduced before being passed to the extractors.}
	{Because extraction is costly, the filtering of artifacts must happen on a file basis before the extractors are called.}
\end{req}

\clearpage
\begin{req}[Configuration of Filters for Input-Files] \label{req:optimization}
\reqtable
	{Must}  {Initial Description}
	{
	The implemented infrastructure must provide means to configure different filters for the files that are to be analyzed.
    }
	{Depending on the extractor in use and the way it processes input files, different options cover different use cases. Furthermore different filters may affect the overall performance of the software system.}

\begin{subreq}[Implementation of Filters for Input-Files]
    \reqtable
	{Must}  {Interview, Initial Description}
	{
	The filters listed below must be implemented (see REQ \ref{req:optimization}):
	\begin{itemize}
		\item \texttt{off} \\
		no filtering 
	    \item \texttt{change only} \\
	    only work with files that were modified
	    \item \texttt{variability-change only} \\
	     only work with files where the variability information was changed
	\end{itemize}
    }
	{In order to evaluate different filter methods and their effect on performance, different categories of filtering are required. \texttt{off} represents the state where no filtering is done, while \texttt{change only} performs superficial filtering without regarding variability information directly. Finally \texttt{variability-change only} is the most sophisticated filtering option of the three as it requires an analysis of the filecontent of each artifact itself.}
\end{subreq}

\begin{subreq}[Implementation of Effect-Filters for Input-Files]
    \reqtable
	{May}  {Interview, Initial Description}
	{
	Of the options listed below must be implemented \cite{req:optimization}:
	\begin{itemize}
		\item \texttt{change-effect} \\
		work with files that were modified and files that were indirectly affected by the change (eg. through includes)
	    \item \texttt{variability-effect}  \\
	    work with files where the variability information was changed and files that were indirectly affected by the variability change
	\end{itemize}
    }
	{When a file is modified, it may affect other files as well that were not changed directly. Consequently, depending on the extractors and analyses used it may be necessary to analyze those files as well.}
\end{subreq}

\end{req}

\begin{req}[Rollback must be possible] \label{req:commit-hook}
\reqtable
	{Must}  {Interview}
	{It must be possible to revert back to the previous state after execution of an incremental analysis.}
	{An Analysis may change the source-files used as input for the extractors along with other changes. Those changes reflect the changes introduced by the new increment that was used as input. If that increment however contains defects identified by the analysis the user might choose to revert the changes and propose an alternative change. This alternative change can only be analyzed if KernelHaven reverts to the original state before the initial change.}

\end{req}

\begin{req}[Dead Code Analysis] 
\reqtable
	{Must}  {Initial Description}
	{The Dead Code Analysis must be integrated in the software system including support for incremental processing}
	{The Dead Code Analyis serves as a practical example for block based analyses. It is used to demonstrate and evaluate the effectiveness of the incremental approach.}
\end{req}

\clearpage
\begin{req}[Target Artifacts] 
\reqtable
    {Must}  {Initial Description}
	{The software system must support the processing of *.c, *.h, makefile, Kbuild and Kconfig files}
	{The filetypes listed above represent the CodeModel, BuildModel and VariabilityModel of the Linux kernel. As the Linux kernel is used for the evaluation of the implemented incremental analysis approach, those filetypes need to be considered.}
\end{req}

\begin{req}[Analysis Result] 
\reqtable
    {May}  {Interview}
	{The analyis result should only contain elements that resulted of the latest analysis}
	{As the analysis should provide feedback for the developer on his work, parts of the software system which could not have been affected by the changes he introduced do not need to be represented in the result.}
\end{req}

\begin{req}[Run existing KernelHaven-Analyses incrementally] 
\reqtable
    {May}  {Interview \\ \emph{Note: This requirement emerged after the implementation choice to use the KernelHaven infrastructure}}
	{It should be possible to run existing analyses incrementally without reimplementation.}
	{Existing analysis plugins for KernelHaven have been written as non-incremental analyses. The adaptation of existing analyses removes the need to reimplement existing analyses. Furthermore the adaptation of existing plugins allows for performance evaluation of incremental analyses against non-incremental analyses.}
\end{req}

\newpage

\section{Implementation of Incremental Analyses}

This section describes the implemented infrastructure for incremental analyses and provides reasoning for the decisions made when engineering the infrastructure. The term \"incremental analyses\" will henceforth be used as shorthand for  \"block based incremental analyses\" as other types of analyses are not covered by the infrastructure.

The infrastructure itself is based on KernelHaven \cite{KernelHaven} as KernelHaven together with its plugins provides a working implementation for various software product line analyses. Through KernelHaven's plugin interfaces, it is also possible to implement the infrastructure for incremental analyses as a plugin itself.

By extending an existing infrastructure, we can benefit from an existing codebase as well as future developments within the KernelHaven project. Furthermore future users of the inremental infrastructure can adjust existing traditional non-incremental analyses written for KernelHaven to run as incremental Analyses.

\subsection{}








%\Bibliography
\newpage

\bibliography{sources}


\end{document}