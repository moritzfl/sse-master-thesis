\documentclass[a4paper]{article}

\usepackage[utf8]{inputenc}
\usepackage[T1]{fontenc}
\usepackage{amsmath}
\usepackage{graphicx}
\usepackage{fancyhdr}
\usepackage{tabularx}
\usepackage{hyperref}
\usepackage{etoolbox}% http://ctan.org/pkg/etoolbox
\usepackage{needspace}% http://ctan.org/pkg/needspace

% Für Theoreme (Beispiele, Definitionen, Sätze etc.)
\usepackage{amsthm}

\usepackage[colorinlistoftodos,prependcaption,textsize=tiny]{todonotes}

%\Bearbeitet Format, sodass statt der Einrückung bei Paragraphen durch Zeilenabstand ersetzt wird
\usepackage{parskip}
\setlength\parindent{0pt}

%\Setzt Dokumentinformationen
\title{Master-Thesis: Incremental Analysis of Software Product Lines}
\author{Moritz Fl\"oter}
\date{June 2018}



\begin{document}
\bibliographystyle{unsrt}

\newtheoremstyle{mystyle}% name
  {3pt}%Space above
  {3pt}%Space below
  {\normalfont}%Body font
  {0pt}%Indent amount
  {\bfseries}% Theorem head font
  {}%Punctuation after theorem head
  {\newline}%Space after theorem head 2
  {}%Theorem head spec (can be left empty, meaning 'normal')

\theoremstyle{mystyle}


\newtheorem{req}{REQ}
\newtheorem{subreq}{REQ}[req]
\newtheorem{subsubreq}{REQ}[subreq]
\AtBeginEnvironment{req}{\Needspace{5\baselineskip}}
\setcounter{req}{0}

\newcommand*{\reqtable}[4]{
\begin{tabular}{ | p{0.15\textwidth} | p{0.79\textwidth} | }
	\hline
	\textit{Priority} & \begin{minipage}[l]{0.79\textwidth}
	\vspace{0.25em}
		#1
	\vspace{0.25em}
	\end{minipage} \\ \hline
	\textit{Source} & \begin{minipage}[l]{0.79\textwidth}
	\vspace{0.25em}
		#2
	\vspace{0.25em}
	\end{minipage}\\ \hline
	\textit{Description} & \begin{minipage}[l]{0.79\textwidth}
	\vspace{0.25em}
		#3
	\vspace{0.25em}
	\end{minipage} \\ \hline
	\textit{Explanation} & \begin{minipage}[l]{0.79\textwidth} 
	\vspace{0.25em}
		#4
	\vspace{0.25em}
	\end{minipage} \\
	\hline
\end{tabular}
}



%\Deckblatt
\maketitle
\newblock

\begin{center}
floeter@uni-hildesheim.de \par
Matrikel-Nr: 236278 \par
Supervisor: \par
Prof. Dr. Klaus Schmid \par
Christian K\"oher
\end{center}

\newpage
\lhead{{}}
\rhead{\leftmark}
\pagestyle{fancy}

\listoftodos[Notes]
\clearpage

%\Inhaltsverzeichnis
\tableofcontents
\newpage
\bibliographystyle{unsrt}

%\Deckblatt
\maketitle
\newpage

\setcounter{page}{1}
\lhead{{}}
\rhead{\leftmark}
\pagestyle{fancy}



\section{Introduction}

\todo{Nicht vergessen: Aufbau der Arbeit erneut beschreiben}

\section{Background}
\subsection{Software Product Lines}

\subsection{Software Product Line Analyses}

\subsection{KernelHaven - an Analysis Infrastructure}\label{kernelhaven}

\todo{describe extension possiblities - plugins, preparation task etc.}

\todo{describe phases in non-incremental infrastructure}

\subsection{Objective for this Work}


\clearpage
\section{Requirements for the Software System}

\begin{req}[Working Base of Incremental Analysis]
	\reqtable
	{Must}  {Interview}
	{The incremental analysis must work based on the current state of a repository and a proposed change}
	{The working base of each incremental analysis is the previous state before a commit/code change. The increment is represented by the change introduced.}
	
	\begin{subreq}[Input Format for Incremental Analysis] \label{req:git-diff}
		\reqtable
		{Must}  {Interview}
		{The input must be a git-diff. Furthermore a filebase upon which the diff can be applied is required.}
		{The main input for the analysis is a git-diff. This diff represents a changeset that is to be analysed.}
	\end{subreq}
\end{req}

\begin{req}[Covered Analysis-Types]
	\reqtable
	{Must}  {Initial Description, Interview}
	{The incremental analysis must support block-based analyses.}
	{}
	
	\begin{subreq}[Input Format for Incremental Analysis] \label{req:git-diff}
		\reqtable
		{Must}  {Interview}
		{The input must be a git-diff}
		{The main input for the analysis is a git-diff. This diff represents a changeset that is to be analysed.}
	\end{subreq}
\end{req}


\begin{req}[Filtering of Source Files] \label{req:early-filtering}
\reqtable
	{Must}  {Interview}
	{The set of source files upon which the analysis operates should be reduced before being passed to the extractors.}
	{Because extraction is costly, the filtering of artifacts must happen on a file basis before the extractors are called.}
\end{req}

\clearpage
\begin{req}[Configuration of Filters for Input-Files] \label{req:optimization}
\reqtable
	{Must}  {Initial Description}
	{
	The implemented infrastructure must provide means to configure different filters for the files that are to be analyzed.
    }
	{Depending on the extractor in use and the way it processes input files, different options cover different use cases. Furthermore different filters may affect the overall performance of the software system.}

\begin{subreq}[Implementation of Filters for Input-Files]
    \reqtable
	{Must}  {Interview, initial description}
	{
	The filters listed below must be implemented (see REQ \ref{req:optimization}):
	\begin{itemize}
		\item \texttt{off} \\
		no filtering 
	    \item \texttt{change only} \\
	    only work with files that were modified
	    \item \texttt{variability-change only} \\
	     only work with files where the variability information was changed
	\end{itemize}
    }
	{In order to evaluate different filter methods and their effect on performance, different categories of filtering are required. \texttt{off} represents the state where no filtering is done, while \texttt{change only} performs superficial filtering without regarding variability information directly. Finally \texttt{variability-change only} is the most sophisticated filtering option of the three as it requires an analysis of the filecontent of each artifact itself.}
\end{subreq}

\begin{subreq}[Implementation of Effect-Filters for Input-Files]
    \reqtable
	{May}  {Interview, initial description}
	{
	The filters listed below may be implemented:
	\begin{itemize}
		\item \texttt{change-effect} \\
		work with files that were modified and files that were indirectly affected by the change (eg. through includes)
	    \item \texttt{variability-effect}  \\
	    work with files where the variability information was changed and files that were indirectly affected by the variability change
	\end{itemize}
    }
	{When a file is modified, it may affect other files as well that were not changed directly. Depending on the extractors and analyses used it may be necessary to analyze those files as well.}
\end{subreq}

\end{req}

\begin{req}[Rollback must be possible] \label{req:rollback}
\reqtable
	{Must}  {Interview}
	{It must be possible to revert back to the previous state after execution of an incremental analysis.}
	{An Analysis may change the source-files used as input for the extractors along with other changes. Those changes reflect the changes introduced by the new increment that was used as input. If that increment however contains defects identified by the analysis the user might choose to revert the changes and propose an alternative change. This alternative change can only be analyzed if KernelHaven reverts to the original state before the initial change.}

\end{req}

\begin{req}[Incremental Dead Code Analysis] 
\reqtable
	{Must}  {Initial description}
	{An incremental dead code analysis must be integrated in the software system}
	{The dead code analyis serves as a practical example for block based analyses. It is used to demonstrate and evaluate the effectiveness of the incremental approach.}
	
	

	
	\begin{subreq}[Incremental Dead Code Analysis Result Format] \label{req:format}
    \reqtable
    {Must}  {Interview}
	{The result should be given in the same csv-format as implemented in the existing UnDeadAnalyzer\footnote{\url{https://github.com/KernelHaven/UnDeadAnalyzer}} plugin for Kernelhaven. If no new analysis is required, a message should be printed to the log and no new result-file must be generated as output.}
	{Maintaining the same format as the non-incremental dead code analysis allows for comparability between incremental and non-incremental analyses. In contrast to non incremental analyses there might be situations where no new analysis is required. If that is the case, a message must be written to the log. However no new result-file has to be generated. \\
	\emph{Note: eventhough the csv-format remains unchangend, the contents of incremental and non-incremental result-files might differ (see REQ \ref{req:coverage})}}
	\end{subreq}
	
	\begin{subreq}[Incremental Dead Code Analysis Result Coverage] \label{req:coverage}
    \reqtable
    {Must}  {Interview, feedback from presentation of initial concept}
	{The analyis result should only contain elements that are results of the latest analysis increment}
	{As the analysis should provide feedback for the developer on his work, parts of the software system which could not have been affected by the changes he introduced do not need to be represented in the result. Therefore merging the result of the increment with previous results is not desired.}
	\end{subreq}
\end{req}

\clearpage
\begin{req}[Target Artifacts] 
\reqtable
    {Must}  {Initial Description}
	{The software system must support the processing of *.c, *.h, makefile, Kbuild and Kconfig files}
	{The filetypes listed above represent the CodeModel, BuildModel and VariabilityModel of the Linux kernel. As the Linux kernel is used for the evaluation of the implemented incremental analysis approach, those filetypes need to be considered.}
\end{req}





\begin{req}[Run existing KernelHaven-Analyses incrementally] 
\reqtable
    {May}  {Interview \\ \emph{Note: This requirement emerged after the implementation choice to use the KernelHaven infrastructure}}
	{It should be possible to run existing analyses incrementally without reimplementation.}
	{Existing analysis plugins for KernelHaven have been written as non-incremental analyses. The adaptation of existing analyses removes the need to reimplement existing analyses. Furthermore the adaptation of existing plugins allows for performance evaluation of incremental analyses against non-incremental analyses.}
\end{req}

\newpage

\section{Implementation of Incremental Analyses}

This section describes the implemented infrastructure for incremental analyses and provides reasoning for the decisions made when engineering the infrastructure. The term 'incremental analyses' will henceforth be used as shorthand for  'block based incremental analyses' as other types of analyses are not targeteted by the infrastructure.

The infrastructure itself is based on KernelHaven\cite{KernelHaven} as KernelHaven together with its plugins provides a working implementation for various software product line analyses. Through KernelHaven's plugin interfaces, it is also possible to implement the infrastructure for incremental analyses as a plugin itself.

By extending an existing infrastructure like KernelHaven, we can benefit from an existing codebase as well as future developments within the KernelHaven project. Furthermore future users of the incremental infrastructure can adjust existing non-incremental analyses written for KernelHaven to run as incremental analyses.

\subsection{Overview: Four Phases of an Incremental Analysis}

As described in \autoref{kernelhaven}

\subsubsection{Tasks of the Preparation-Phase}

The goals of the preparation-phase in the incremental infrastructure are to apply relevant changes to the codebase and reduce the number of files that the extraction has to run on.

KernelHaven usually triggers extractors when an analysis is initialized. This allows for clean analysis implementations without direct calls to extractors as the analysis is directly provided with the extraction result. Therefore the extraction process is not controlled by the analysis itself - an analysis implementations just expects models as inputs and does not need to care for where they come from. Instead, the KernelHaven infrastructure provides the models for the analysis.

In this scenario, the extraction is executed based on parameters defined in a configuration-file that gets passed to KernelHaven as a command line parameter.

In our incremental analyses infrastructure, there steps that prepare the input and configuration for the extractors before they are started.

For an incremental analysis the extraction of the models can not simply run on the entire codebase as REQ \ref{req:early-filtering} requires that input files are filtered before extraction. This saves computational effort when running the extractors as some parts of the codebase  may not need to be processed again. Instead the result of previous extractions may be reused.

 But even before any filtering on the artifacts in the codebase can happen, the incremental infrastructure needs to ensure a codebase that represents the increment that is to be analyzed. The main input is provided through a git-diff file describing changes to the codebase REQ \ref{req:git-diff}. 
 
 In order to allow for reuse of existing extractors, the input must be processed in a way that they can work with. As existing extractors work based on the access to the filesystem, the changes in the diff-file must be merged with the codebase.

After integrating the changes into the codebase the extractors could technically run in the same manner as they do within an non-incremental analysis. However because we assume that the previous state of the codebase has already been extracted in a previous run when running an incremental analysis, the extractors only need to extract information from a subset of files. The remaining extraction results may be reused from the previous run of the extractors.

\subsubsection{Tasks of the Post-Extraction-Phase}

Because of the filtering performed in the Preparation-Phase, the analysis may not simply run on the result of the extractors directly. Therefore it is merged with the results of previous extractions. After merging, it passes a representation of the model to the analysis.

While merging the extractors outputs, it is important that no prior extraction results gets directly overwritten. This is because REQ \ref{req:rollback} requires that the state before the execution of an incremental analysis can be restored. Therefore the Post-Extraction-Phase uses a modified implementation of the cache-system, that the main infrastructure of KernelHaven uses. This \texttt{HybridCache} allows storage and access to two versions of the modelset. It guarantees that the previous modelset can be restored.

While a rollback to previous extraction-results is the main reason for introducing the \texttt{HybridCache}, a side benefit is that analyses now may use both the current and the previous state of the codebase. \todo{eventually discuss this in more detail? or reference to a later chapter}.

As 


\subsection{The Preparation-Phase}

\subsubsection{Technical Implementation}

Because the definition of inputs for the extractors of the Preparation-Component needs to happen before the extraction and analysis itself, it makes sense to run them before an analysis is instantiated. With the \texttt{IPreparation}-interface, KernelHaven offers a mechanism to execute code before the rest of the infrastructure is started. Therefore it allows to modify the configuration that is initially described by a configuration-file. Furthermore other tasks independent of the modifications to the configuration such as applying changes to the codebase may also be executed.

For the incremental infrastructure, the \texttt{IncrementalPreparation}-class implements the \texttt{IPreparation}-interface and merges the changes described by the diff-file with the exisiting codebase. It then continues to  modify the configuration by defining the actual input-files upon which the extractors are run.

Using the \texttt{IPreparation}-interface preserves the usual data-flow within KernelHaven where extractors run based on a configuration-file and pass their results to the analyses. An alternative would have been to implement those tasks as an \texttt{AnalysisComponent} or \texttt{IAnalysis} both of which represent possibilies to create an analysis plugin for KernelHaven. This would however represent a breach in the data-flow of KernelHaven as an analysis would now directly define the inputs for the extractors.

Furthermore, the the \texttt{IPreparation}-interface offers more flexibility to adjust the configuration of KernelHaven as all parts of the infrastructure - even the actual analysis-class itself - can be configured from within an \texttt{IPreparation}-implementation. Therfore the modifiability of the incremental infrastructure is higher than with analysis-classes.



%\Bibliography
\newpage

\bibliography{sources}


\end{document} 